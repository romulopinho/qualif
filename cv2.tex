\documentclass[10pt]{article}

\usepackage[sectionbib]{bibunits}
\usepackage{fullpage}
\usepackage[latin1]{inputenc}
\usepackage[OT1]{fontenc}
\usepackage{sectsty}
\usepackage{titlesec}

\sectionfont{\large}
\paragraphfont{\normalsize}
\titlespacing*{\section}{0mm}{*2}{*0.5}

\newenvironment{itemize*}%
  {\begin{itemize}%
    \setlength{\itemsep}{0pt}%
    \setlength{\parskip}{1mm}}%
  {\end{itemize}}

\title{Curriculum Vit\ae}
\date{}
\author{}

\begin{document}

\maketitle
\vspace{-1,5cm}
\flushleft{{\bf R�mulo (TEIXEIRA DE ABREU) PINHO} \\
\medskip
40, Rue Colin, Villeurbanne, 69100, France \\
Mobile: +33-(0)-638305900 \\
E-mail: romulo.pinho@lyon.unicancer.fr\\
URL: www.creatis.insa-lyon.fr/rio/RomuloPinho} \\

\medskip

Br�silien, 34 ans \\
Mari�, sans enfants

\medskip

\section{Formation}

\begin{itemize*}
\item Ph.D. en Physique, Novembre, 2010\\
Universeit Antwerpen, Anvers, Belgique\\
Sp�cialisation: Traitement d'images m�dicales \\
Th�se: A Decision Support System for the Assessment and Stenting of Tracheal Stenosis (en anglais). \\

\item M.Sc. en Ing�nierie de Syst�mes, Mai, 2001\\
Universidade Federal do Rio de Janeiro, Rio de Janeiro, Br�sil\\
Sp�cialisation: Computation Graphique\\
Th�se: Three Dimensional Reconstruction of Head Models from Magnetic Resonance Images (en portugais). \\

\item B.Sc. en Science de la Computation, Septembre, 1998\\
Universidade Federal Fluminense, Niter\'oi, Br�sil\\
\end{itemize*}

%\section{Certificats}
%
%\begin{itemize*}
%	\item ScrumMaster (Las Vegas, avril, 2005).
%\end{itemize*}

%\pagebreak

\section{Langues}

\begin{itemize*}
	\item portugais - maternelle
	\item anglais - courant 
	\item fran�ais - niveau interm�diaire
	\item n�erlandais - niveau basique
\end{itemize*}

%%\renewcommand{\refname}{Contributions}
%\begin{bibunit}[unsrt]
%\bf{Book Chapters} \\
%\nocite{Pinho:Cache1}
%\nocite{Pinho:Cache2}
%\bf{Journals} \\
%\nocite{Pinho:Airways1}
%\nocite{Pinho:Airways2}
%\bf{Conference Papers (full paper)} \\
%\nocite{Pinho:Trachea1}
%\nocite{Pinho:Trachea2}
%\nocite{Pinho:Trachea3}
%\nocite{Pinho:Trachea4}
%\nocite{Pinho:Trachea5}
%\nocite{Pinho:Trachea6}
%\nocite{Pinho:Trachea7}
%\putbib[mybib]
%\end{bibunit}

%\bibliographyunit[\section]
\section{Publications Scientifiques}

%\renewcommand{\refname}{Journal papers}
\renewcommand{\refname}{Revues}
\begin{bibunit}[unsrt]
\nocite{Pinho:Trachea7}
\nocite{Pinho:Trachea4}
%\nocite{Pinho:Cache2}
\putbib[mybib]
\end{bibunit}

%\renewcommand{\refname}{Book chapters}
\renewcommand{\refname}{Chapitres de Livres}
\begin{bibunit}[unsrt]
\nocite{Pinho:Trachea5}
\putbib[mybib]
\end{bibunit}

%\renewcommand{\refname}{Conference proceedings (full paper)}
\renewcommand{\refname}{Conf�rences Internationales avec Comit� et Publication des Actes}
\begin{bibunit}[unsrt]
\nocite{DELM-11}
\nocite{PINH-11}
\nocite{RIT-11b}
\nocite{Pinho:Trachea6}
\nocite{Pinho:Airways2}
\nocite{Pinho:Trachea3}
\nocite{Pinho:Trachea2}
\nocite{Pinho:Cache1}
\nocite{Pinho:Trachea1}
\nocite{Pinho:Airways1}
\putbib[mybib]
\end{bibunit}

\renewcommand{\refname}{Conf�rences Internationales avec Comit� et Publication des Actes (r�sum�)}
\begin{bibunit}[unsrt]
\nocite{Bernat:Clavicle}
\nocite{Huysmans:Clavicle}
\nocite{Pinho:Trachea8}
\putbib[mybib]
\end{bibunit}


\section{Parcours Acad�mique}
\paragraph{Enseignement}

\begin{itemize*}
\item Vacataire\\
Syst�mes d'Exploitation (en fran�ais, TP: 16h)\\
INSA-Lyon\\
Septembre, 2011 -- pr�sent.

\item Vacataire\\
Programmation Orient�e � l'Objet (en anglais, TD + TP: 45,5h eq. TD)\\
INSA-Lyon\\
Septembre, 2011 -- pr�sent.
\end{itemize*}

\paragraph{Supervision de Th�ses}

\begin{itemize*}
\item Projet de licence de Stijn Peeters (co-supervision)\\
Automatic segmentation of the airways: Morphological study of the trachea (en n�erlandais)\\
Universiteit Antwerpen, 2009

\item Projet de licence de Sten Luyckx (co-supervision)\\
Automatic segmentation of the airways in 3D CT using cylindrical shape modelling (en n�erlandais)\\
Universiteit Antwerpen, 2009 
\end{itemize*}

\paragraph{Projets de Recherche}
\begin{itemize*}
\item CIMI -- IBBT (http://www.ibbt.be/en/projects/overview-projects/p/detail/cimi-2) \\
Universiteit Antwerpen \\
Responsable et d�v�loppeur principal du Task 2.1 du Work Package 3 \\
Janvier, 2010 -- D�cembre, 2010

\item Segmentor -- IBBT \\
Universiteit Antwerpen \\
Chercheur \\
Septembre, 2009 -- D�cembre, 2010

\end{itemize*}

\section{Parcours Professionnel}

\begin{itemize*}
\item Ing�nieur de Recherche (post-doc)\\
Centre L�on B�rard, Lyon, France\\
Janvier, 2011 -- pr�sent.

\item Ing�nieur de syst�mes de TV/Chercheur\\
Dept. R\&D, TV Globo Ltd., Rio de Janeiro, Br�sil\\
Janvier, 2001 -- Avril, 2006.

\item Ing�nieur de Logiciels/Chercheur\\
LAMEC (Laboratoire d'El�ments de Contour) \\
Universidade Federal do Rio de Janeiro, Rio de Janeiro, Br�sil\\
Octobre, 2001 -- Mai, 2003.

\item Ing�nieur de Logiciels \\
Publintel S. A., Rio de Janeiro, Br�sil\\
Aout, 2000 -- D�cembre, 2000.

\item Chercheur/D�veloppeur\\
Montreal Inform\`atica, Rio de Janeiro, Br�sil \\
Mars, 2000 -- Aout, 2000.

\item Analyste de Syst�mes/Programmeur\\
Delta de Friburgo, Rio de Janeiro, Br�sil\\
Janvier, 2000 -- Mars, 2000

\item Programmeur Leader \\
Brainstorming Ass. e Plan. em Inform\`atica\\
Avril, 1999 -- Janvier, 2000

\item Programmeur de Jeux Vid�os\\
Z-Movie Studio\\
Janvier, 1998 -- Janvier, 1999

\item Programmeur\\
ADD-Labs � Universidade Federal Fluminense\\
Octobre, 1996 -- Janvier, 1998

\end{itemize*}

%\section{Other Relevant Professional Experience}

%\begin{itemize*}
%	\item 10-year experience working with C/C++;
%	\item 4-year experience working with OpenGL;
%	\item 3-year experience working with DirectShow
%	(filter user experience and basic filter development
%	experience � DirectX 7.0, 8.0, 8.1);
%	\item 2-year experience working with Delphi;
%	\item 3-year experience working with ORACLE, MySQL
%	and PL/SQL;
%	\item 1-year experience working with DirectDraw and
%	Direct3D (DirectX 6.0);
%	\item Good knowledge of MFC, STL and general Win32
%	programming;
%	\item Good knowledge of Linux programming;
%\end{itemize*}


\end{document}