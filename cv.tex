\documentclass{letter}

\begin{document}


\begin{center}
\textbf{Romulo Teixeira de Abreu Pinho\\Curriculum Vitae\\}
\end{center}


\textbf{Home Address}

101 bis, Cours Fauriel\\
Sain Etienne, 42100, France.\\
Cel. Number: +33-(0)-638305900\\
e-mail: romulopinho@yahoo.com.br

\textbf{Office Address}

Vision Lab\\ 
University of Antwerp (CMI)\\  
Universiteitsplein 1, N.1.18\\
Wilrijk, Antwerp, Belgium, B-2610\\ 

\textbf{Biographical Data}

Birth date: March, 3rd 1977.\\
Place of Birth: Rio de Janeiro, Brazil.\\
Citizenship: Brazilian.\\
Marital status: Married.

\textbf{Professional Interests}

\begin{itemize}
	\item Continuous engagement with research activities;
	\item Continuous contact with cutting-edge technology;
	\item Computer Graphics - Medical Imaging, Animation, Game Programming, 3d Modeling, Computer Geometry; 
	\item C++ and C Programming;
	\item Multimedia Development;
	\item Development of complex and performance critical software;
	\item High and low level programming;
	\item OOP.
\end{itemize}

\textbf{Education}

Ph.D. student in Physics. Since April, 2006.\\
University of Antwerp, Belgium.\\
Specialization: Medical Image Segmentation.\\
\\
M.Sc. degree in Software Engineering. May, 2001.\\
Federal University of Rio de Janeiro, Rio de Janeiro, RJ, Brazil.\\
Specialization: Computer Graphics.\\
Thesis: Three Dimensional Reconstruction of Head Models from Magnetic Resonance Images.\\
\\
B.Sc. degree in Computer Science. September, 1998.\\
Fluminense Federal University, Niteroi, RJ, Brazil.\\
Specialization: Computer Science.

\textbf{Honors, Awards and Certificates}

\begin{itemize}
	\item ScrumMaster Certificate (Las Vegas, April,2005)
	\item Awarded the Certificate in Advanced English (CAE) from the University of Cambridge (December, 1996).
\end{itemize}

\textbf{Professional Experience}

\begin{enumerate}
\item TV Systems Software Engineer/Researcher\\
Company: R\&D Department, TV Globo Ltd.\\
Period: January, 2001 -- April, 2006.\\

Design and development of software for general acquisition,
compression, storage, processing and transmission of
video and audio signals. The main project I worked
on is a Store \& Forward system which enables TV
correspondents to send the journalistic material they
tape to the Company�s HQs in Rio de Jeneiro and Sao
Paulo over the Internet. A Win32 client software
captures, compresses, encrypts and transfers the video
to a Linux sever software using a proprietary FTP-like
protocol. Another Win32 video server decompresses the
received video and prepares it for play-out.
Applications are implemented in C/C++ for Win32 and
Linux platforms, using various SDKs and libraries
including DirectShow SDK, Windows Media SDK, Matrox
DigiSuite SDK, MFC, Video4Linux, MySQL.\\


\item Software Engineer/Researcher\\
Company: LAMEC �- Boundary Elements Method Laboratory,\\
Federal University of Rio de Janeiro.\\
Period: October, 2001 �- May, 2003.\\

Design and development of a C++, OpenGL/QT-based scientific
visualization and geometric modeling system to build 3d 
models out of simple 3d primitives. The models were then
color mapped according to numerical results obtained from
other applications.\\

\item Software Engineer\\
Company: Publintel S. A.\\
Period: August, 2000 �- December, 2000.\\

Development and maintenance of GIS applications. Translation
of existing GUI code to the Win32 environment. Design and
implementation of a logistics database for automatic vehicle 
localization, using SQL-Server, C++, STL, MFC and ADO.\\

\item Technology Researcher/Programmer\\
Company: Montreal Inform�tica\\
Period: March, 2000 �- August, 2000.\\

I was hired to start a project on Speech Recognition.
My duties were to search for current and new
technologies in this area, in order to use an existing 
API for developing voice interpretation systems. I 
also worked with biometrics, in an AFIS (Automatic 
Fingerprint Identification System) project. These 
applications were implemented in C++ and Delphi, using 
the ORACLE DBMS.\\

\item System Analyst/Programmer\\
Company: Delta de Friburgo\\
Period: January, 2000 �- March, 2000.\\

I took part in the design of a profit optimization
system for the Brazilian Petroleum Company
(PETROBRAS), using C++ and the ORACLE DBMS\\

\item Programmer\\
Company: Brainstorming Ass. e Plan. em Inform�tica\\
Period: April, 1999 �- January, 2000.\\

Lead programmer of a database system in Delphi/ORACLE for controlling 
the use of the necessary material for the repair of ships, weapons 
and for general services inside a Brazilian Navy base in Rio de Janeiro.\\

\item Game Programmer\\
Company: Z-Movie Studio.\\
Period: January, 1998 �- January, 1999.\\

Development of 3D educational computer games. Included design and
implementation of 3D engine, character animation, GUIs and game logic.
Implementations in C++ for DirectX 6.0.\\

\item Programmer\\
Company: ADD-Labs -- Universidade Federal Fluminense (Laboratory of Active Document Design)\\
Period: October, 1996 �- January, 1998.\\

Development of a prototype Virtual Reality system built in C/C++ to the Brazilian Petroleum 
Company (PETROBRAS). The objective was to provide the user with a 3D 
visualization of oil extraction fields. Results were published in a science seminar 
promoted by the University.

\end{enumerate}

\textbf{Other Relevant Professional Experience}

\begin{itemize}
	\item 7-year experience working with C/C++;
	\item 3-year experience working with OpenGL;
	\item 3-year experience working with DirectShow
	(filter user experience and basic filter development
	experience � DirectX 7.0, 8.0, 8.1);
	\item 2-year experience working with Delphi;
	\item 3-year experience working with ORACLE, MySQL
	and PL/SQL;
	\item 1-year experience working with DirectDraw and
	Direct3D (DirectX 6.0);
	\item Good knowledge of MFC, STL and general Win32
	programming;
	\item Good knowledge of Linux programming;
\end{itemize}

\textbf{Conferences and Workshops}

\begin{itemize}
	\item News Technologies implemented in Iraq War,
	SET Conference, 2003
	\item Introductory Level Lecture on Computer
	Graphics, 2003
	\item Introductory Level Lecture on Computer
	Graphics, 2002
	\item Intel do Brasil � Launch of the Intel Pentium
	III processor. Intel�s partner in the development of
	Pentium III optimized software. In this workshop I
	took part in the presentation of one of game products
	cited above, 1998;
\end{itemize}

\end{document}