\documentclass{article}

\usepackage[subsectionbib]{bibunits}
\usepackage{fullpage}
%\usepackage[sort]{natbib}

\title{R\^omulo Teixeira de Abreu Pinho\\Curriculum Vitae}
\date{}
\author{}

\begin{document}

\maketitle


\section{Contact Information}

40, Rue Colin\\
Villeurbanne, 69100, France.\\
Mobile Number: +33-(0)-638305900\\
e-mail: romulo.pinho@lyon.unicancer.fr\\
URL: www.creatis.insa-lyon.fr/rio/RomuloPinho

\section{Biographical Data}

Birth date: March, 3rd, 1977\\
Place of Birth: Petr\'opolis, RJ, Brazil\\
Citizenship: Brazilian\\
Marital status: Married

%\section{Professional Interests}

%\begin{itemize}
%	\item Research and Development;
%	\item Computer Graphics - Medical Imaging, Animation, Game Programming, 3d Modelling, Computer Geometry; 
%	\item Multimedia Development;
%	\item C++ and C Programming;
%\end{itemize}

\section{Education}

\begin{itemize}
\item Ph.D. in Physics. November, 2010\\
University of Antwerp, Belgium\\
Specialization: Medical Image Processing\\
Thesis: A Decision Support System for the Assessment and Stenting of Tracheal Stenosis.

\item M.Sc. in Software Engineering. May, 2001\\
Federal University of Rio de Janeiro, Rio de Janeiro, RJ, Brazil\\
Specialization: Computer Graphics\\
Thesis: Three Dimensional Reconstruction of Head Models from Magnetic Resonance Images.

\item B.Sc.  in Computer Science. September, 1998\\
Fluminense Federal University, Niter\'oi, RJ, Brazil\\
Specialization: Computer Science.
\end{itemize}

\section{Honours, Awards and Certificates}

\begin{itemize}
	\item ScrumMaster Certificate (Las Vegas, April, 2005).
\end{itemize}

\pagebreak

\section{Languages}

\begin{itemize}
	\item Portuguese - native
	\item English - fluent (non-native)
	\item French - intermediate level
	\item Dutch - basic level
\end{itemize}

%\renewcommand{\refname}{Contributions}
%\begin{bibunit}[unsrt]
%\bf{Book Chapters} \\
%\nocite{Pinho:Cache1}
%\nocite{Pinho:Cache2}
%\bf{Journals} \\
%\nocite{Pinho:Airways1}
%\nocite{Pinho:Airways2}
%\bf{Conference Papers (full paper)} \\
%\nocite{Pinho:Trachea1}
%\nocite{Pinho:Trachea2}
%\nocite{Pinho:Trachea3}
%\nocite{Pinho:Trachea4}
%\nocite{Pinho:Trachea5}
%\nocite{Pinho:Trachea6}
%\nocite{Pinho:Trachea7}
%\putbib[mybib]
%\end{bibunit}

%\bibliographyunit[\section]
\section{Publications Scientifiques}

%\renewcommand{\refname}{Journal papers}
\renewcommand{\refname}{Revues}
\begin{bibunit}[unsrt]
\nocite{Pinho:Trachea7}
\nocite{Pinho:Trachea4}
%\nocite{Pinho:Cache2}
\putbib[mybib]
\end{bibunit}

%\renewcommand{\refname}{Book chapters}
\renewcommand{\refname}{Chapitres de Livres}
\begin{bibunit}[unsrt]
\nocite{Pinho:Trachea5}
\putbib[mybib]
\end{bibunit}

%\renewcommand{\refname}{Conference proceedings (full paper)}
\renewcommand{\refname}{Conf�rences Internationales avec Comit� et Publication des Actes}
\begin{bibunit}[unsrt]
\nocite{DELM-11}
\nocite{PINH-11}
\nocite{RIT-11b}
\nocite{Pinho:Trachea6}
\nocite{Pinho:Airways2}
\nocite{Pinho:Trachea3}
\nocite{Pinho:Trachea2}
\nocite{Pinho:Cache1}
\nocite{Pinho:Trachea1}
\nocite{Pinho:Airways1}
\putbib[mybib]
\end{bibunit}

\renewcommand{\refname}{Conf�rences Internationales avec Comit� et Publication des Actes (r�sum�)}
\begin{bibunit}[unsrt]
\nocite{Bernat:Clavicle}
\nocite{Huysmans:Clavicle}
\nocite{Pinho:Trachea8}
\putbib[mybib]
\end{bibunit}


\section{Academic Experience}

\subsection{Teaching}
\begin{itemize}
\item Teacher assistant\\
Subject: Operating Systems (in French, 16h, practice sessions)\\
Institution: INSA-Lyon\\
Period: September, 2011 -- February, 2012.

\item Teacher assistant\\
Subject: Object Oriented Programming (in English, 45h, theory and practice sessions)\\
Institution: INSA-Lyon\\
Period: September, 2011 -- February, 2012.
\end{itemize}

\subsection{Supervision of Theses}
\begin{itemize}
\item B.Sc. Thesis of Stijn Peeters (co-supervision)\\
Title: Automatic segmentation of the airways: Morphological study of the trachea (in Dutch)\\
Institution: University of Antwerp\\
Period: January, 2009 -- June, 2009.

\item B.Sc. Thesis of Sten Luyckx (co-supervision)\\
Title: Automatic segmentation of the airways in 3D CT using cylindrical shape modelling (in Dutch)\\
Institution: University of Antwerp\\
Period: January, 2009 -- June, 2009.
\end{itemize}

\subsection{Research Projects}
\begin{itemize}
\item CIMI -- IBBT (http://www.ibbt.be/en/projects/overview-projects/p/detail/cimi-2) \\
Period: January, 2010 -- December, 2010 \\
Role: Responsible for and lead researcher of Task 2.1 of Work Package 3

Development of cache and pre-fetching techniques for out-of-core processing and visualization of large microscopic images. 

\item Segmentor -- IBBT \\
Period: September, 2009 -- December, 2010 \\
Role: Researcher

Development of semi-automatic segmentation techniques based on haptics for the segmentation of tracheal stenosis.

\end{itemize}


\section{Professional Experience}

\begin{itemize}
\item Post-doc Research Engineer\\
Company: Centre L\'eon B\'erard, Lyon, France\\
Period: Since January, 2011

Design, implementation, and deployment of a medical imaging system for image guided
radiation therapy against lung cancer. The aim is to carry out a clinical trial for the use of new
image registration techniques and their influence on radiation therapy against moving tumours. 
Multi-platform development using C++, CMake, ITK, VTK, QT, Python, and shell scripts.

\item TV Systems Software Engineer/Researcher\\
Company: R\&D Department, TV Globo Ltd.\\
Period: January, 2001 -- April, 2006

Design and development of software for general acquisition,
compression, storage, processing, and transmission of
video and audio signals. Applications were implemented in C/C++ for Win32 and
Linux platforms, using various SDKs and libraries
including DirectShow SDK, Windows Media SDK, Matrox
DigiSuite SDK, MFC, Video4Linux, MySQL.

\item Software Engineer/Researcher\\
Company: LAMEC �- Boundary Elements Method Laboratory,\\
Federal University of Rio de Janeiro.\\
Period: October, 2001 �- May, 2003

Design and development of a C++, OpenGL/QT-based scientific
visualization and geometric modelling system to build 3d 
models out of simple 3d primitives, to be used in numerical computations.

\item Software Engineer\\
Company: Publintel S. A.\\
Period: August, 2000 �- December, 2000

Development and maintenance of GIS applications. Translation
of existing GUI code to the Win32 environment. Design and
implementation of a logistics database for automatic vehicle 
localization, using SQL-Server, C++, STL, MFC, and ADO.

\item Technology Researcher/Programmer\\
Company: Montreal Inform\`atica\\
Period: March, 2000 �- August, 2000

Design and implementation of Speech Recognition and
biometrics (fingerprint identification) systems. These 
applications were implemented in C++ and Delphi, using 
the ORACLE DBMS.

\item System Analyst/Programmer\\
Company: Delta de Friburgo\\
Period: January, 2000 �- March, 2000

I took part in the design of a profit optimization
system for the Brazilian Petroleum Company
(PETROBRAS), using C++ and the ORACLE DBMS

\item Programmer\\
Company: Brainstorming Ass. e Plan. em Inform\`atica\\
Period: April, 1999 �- January, 2000

Lead programmer of a database system to control 
the use of the necessary material for the repair of ships, weapons 
and for general services inside a Brazilian Navy base in Rio de Janeiro.
Development in Delphi/ORACLE.

\item Game Programmer\\
Company: Z-Movie Studio.\\
Period: January, 1998 �- January, 1999

Development of 3D educational computer games. Design and
implementation of 3D engine, character animation, GUIs and game logic.
Implementations in C++ for DirectX 6.0.

\item Programmer\\
Company: ADD-Labs -- Universidade Federal Fluminense (Laboratory of Active Document Design)\\
Period: October, 1996 �- January, 1998

Development of a prototype Virtual Reality system built in C/C++ to the Brazilian Petroleum 
Company (PETROBRAS). The objective was to provide the user with a 3D 
visualization of oil extraction fields.

\end{itemize}

%\section{Other Relevant Professional Experience}

%\begin{itemize}
%	\item 10-year experience working with C/C++;
%	\item 4-year experience working with OpenGL;
%	\item 3-year experience working with DirectShow
%	(filter user experience and basic filter development
%	experience � DirectX 7.0, 8.0, 8.1);
%	\item 2-year experience working with Delphi;
%	\item 3-year experience working with ORACLE, MySQL
%	and PL/SQL;
%	\item 1-year experience working with DirectDraw and
%	Direct3D (DirectX 6.0);
%	\item Good knowledge of MFC, STL and general Win32
%	programming;
%	\item Good knowledge of Linux programming;
%\end{itemize}


\end{document}